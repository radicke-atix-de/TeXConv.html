\documentclass[10pt,a4paper]{article}
\usepackage[utf8x]{inputenc}
\usepackage{ucs}
\usepackage{amsmath}
\usepackage{amsfonts}
\usepackage{amssymb}
\usepackage{hyperref}
\usepackage{graphicx}
\usepackage{float}
\usepackage{listings}

\usepackage[german]{babel}
\usepackage[T1]{fontenc}

\lstset{language=bash, title=[CODEAUSSCHNITT], numbers=left, basicstyle=\footnotesize}

\newcommand{\doppelt}[2]{\textit{\textbf{#1}} \texttt{#2}}
%\newcommand{\einfach}[1]{\textit{\textbf{#1}}}
\usepackage{fake}


\author{Olaf Radicke}
\title{Einfaches LaTeX-Dokument}

% Ein Kommentar!! mit Lehrzeichen. 

% Kurz vor beginn
\begin{document}
% Kurz nach beginn


\maketitle

\newpage

\tableofcontents

\newpage

\section {Vorwort mit Index}

In der Mathematik drücken Klammern unter anderem einen Vorrang einer
auszuführenden Rechenoperation vor anderen in der Rechenreihenfolge aus. Zum
Beispiel ist das Ergebnis\index{Ergebnis} von 10 − (6 − 1) gleich 5, da
die Rechnung innerhalb\index{innerhalb} %Inline Kommentar.
der Klammer zuerst ausgeführt wird, 10 − 6 − 1 ist dagegen gleich 3, da in
diesem Fall von links nach rechts vorgegangen wird.
Es sind 5\% der der Kommentare fehlerhaft.

\begin{lstlisting}
    # set fstyp-handler
    echo '[ -e $NEWROOT/proc ]' > $hookdir/initqueue/finished/osrroot.sh

    # Handler
    inst_simple "$moddir/osr-detect-root.sh" "/sbin/osrroot"
\end{lstlisting}


In der höheren Mathematik
dienen Klammern auch noch vielen anderen Zwecken, vor allem der Bezeichnung
von Argumenten einer Funktion. Geschweifte, eckige und spitze Klammern haben
in der Mathematik meist eine spezielle Bedeutung.

Das darf nicht interpretiert werden:
\begin{verbatim}
\section {Tex-Imput}

The \\input command causes the indicated file to be read and processed,
exactly as if its contents had been inserted in the current file at that
point. The file name may be a complete file name with extension or just a
first name, in which case the file FILE.TEX is used.
\end{verbatim} 

\section {Tex-Imput}

The \\input command causes the indicated file to be read and processed,
exactly as if its contents had been inserted in the current file at that
point. The file name may be a complete file name with extension or just a
first name, in which case the file FILE.TEX is used.

%\section {Tex-Include}

Noch mal verb-Umgebung:

\verb|\include|

Normal weiter (mit bigskip)...

\bigskip

The \\include command is used in conjunction with the \\includeonly command for
selective inclusion of files. The file argument is the first name of a file,
denoting FILE.TEX. If file is one the file names in the file list of the
\\includeonly command or if there is no \\includeonly command, the \\include
command is equivalent to

\begin{verbatim}
\clearpage \input{file} \clearpage
\end{verbatim} 


except that if the file FILE.TEX does not exist, then a warning message
rather than an error is produced. If the file is not in the file list, the
\\include command is equivalent to \clearpage.

The \\include command may not appear in the preamble or in a file read by
another \\include command.

\section{Hauptteil mit Querverweis}

In der Mathematik werden Klammern ebenfalls meist paarig eingesetzt, wobei
öffnende und schließende Klammer jeweils zueinander spiegelsymmetrisch sind.
Es existieren jedoch Ausnahmen, etwa bei Intervallklammern und auch einzelne,
nicht paarige Klammern werden bisweilen verwendet.

Siehe auch: Kapitel \ref{sec:quell} Seite \pageref{sec:quell}

\subsection{Funktionsargumente mit Auflistung}

Normalerweise werden Argumente von Funktionen in runde Klammern gesetzt,
gelegentlich auch in spitze, um eine bessere Unterscheidbarkeit zu
gruppierenden Klammern zu ermöglichen (f ist eine Funktion, g eine Variable):

\begin{itemize}
 \item \verb|<|moddir\verb|>| /issue nach /etc
 \item \verb|<|moddir\verb|>|/shinit.sh nach /sbin
 \item \verb|<|moddir\verb|>|/lib/boot-lib.sh nach /lib/osr
 \item \verb|<|moddir\verb|>|/lib/defaults.sh nach /lib/osr
 \item \verb|<|moddir\verb|>|/lib/repository-lib.sh nach /lib/osr
 \item \verb|<|moddir\verb|>|/lib/rootfs-lib.sh nach /lib/osr
 \item \verb|<|moddir\verb|>|/lib/shinit.sh nach /lib/osr
 \item \verb|<|moddir\verb|>|/lib/std-lib.sh nach /lib/osr
\end{itemize}

\subsection{Satzzeichen nit Tabelle}

Die durch die Satzzeichen bewirkte Strukturierung hilft, den gewünschten
Sprech- oder Atemrhythmus der gesprochenen Rede auf Papier festzuhalten,
ohne den das Gesprochene wie eine schlechte Automatenstimme klingen würde.

\begin{tabular}{|l|l|l|}
 \hline
\textbf{Hook} & \textbf{Priorität} & \textbf{Script} \\ \hline
netroot  & 11 & osr-detect-chroot.sh \\ \hline
netroot  & 50 & osr-mount-chroot.sh \\ \hline
pre-udev & 51 & osr-move-chroot.sh \\ \hline
\end{tabular}

\subsection{Definition}


\begin{description}
\item[Hook:] emergency
\item[Priorität:] 1
\end{description}

\section{Nachwort mit eigenen Komando}

Beginnent mit \doppelt{Param eins}{Param zwei} und endent mit Drei....

Die heutigen Satzzeichen gehen im Wesentlichen auf den venezianischen Drucker
Aldus Manutius den Älteren (1450–1515) zurück. Der Erfinder der Kursivschrift
(ihrer Herkunft wegen im Englischen als Italics bezeichnet) druckte in Pietro
Bembos 1494 erschienenem Werk De Aetna das erste Semikolon und standardisierte,
gefolgt von seinem gleichnamigen Enkel, die Satzzeichen: Die Virgel begann
damals so auszusehen wie das heutige Komma. Hatte die Zeichensetzung bis dahin
den Zweck, für das Vorlesen aus Büchern Hinweise auf Tonfall und Atempausen zu
geben, so hielt Aldus Manutius der Jüngere 1566 fest, Zweck der Interpunktion
sei hauptsächlich, Klarheit in die Syntax zu bringen.


Die Setzung von Satzzeichen ist oft auch für die Vermittlung des gewünschten
Inhalts wichtig. Die nachfolgende frei erfundene Anekdote verdeutlicht dies:
Vor langer langer Zeit gab es einen Bösewicht, der hingerichtet werden sollte.
Man schickte nach dem König. Er hatte das Recht inne, den Delinquenten zu
begnadigen. Ein Bote kam vom König mit folgender Botschaft zurück: „Ich
komme nicht köpfen!“. Nur, wo sollte man das Komma setzen? „Ich komme, nicht
köpfen!“ oder „Ich komme nicht, köpfen!“? Sogar ohne Komma ergibt der Satz
einen Sinn, auch wenn ein König damit sicher den Befehl an seinen Henker
gemeint hätte.


\section{Quellen- und Literaturangaben}
\label{sec:quell}

\begin{itemize}
 \item fedora-Projekt: \url{http://fedoraproject.org/wiki/Dracut}
 \item Projektseite: \url{https://dracut.wiki.kernel.org/}
 \item News: \url{http://git.kernel.org/?p=boot/dracut/dracut.git;a=blob_plain;f=NEWS}
\end{itemize}


% Kurz vor Ende
\end{document}
% Ende des Dokuments...
